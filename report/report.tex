\documentclass[a4paper, 12pt]{article}
\usepackage{graphicx}
\usepackage{listings}

\setlength\parindent{24pt}

\lstset{language=python,breaklines=true, frame=single}

\begin{document}
\begin{figure}
    \centering
    \includegraphics[width=1\textwidth]{Logo}
\end{figure}

\title{Project Report}
\author{Manwel Bugeja}
\date{\today}
\maketitle
  
\tableofcontents
\newpage

%\section{Introduction}
%This is an intro. \cite{lowhighlevelevents}
%That was a citation.

\section{Pay-to-Public-Key-Hash (P2PK)} % (fold)
\label{sec:pay_to_pubkey_p2pk}

\subsection{Overview} % (fold)
\label{sub:overview}

When using this script, the public key is stored as plaintext within the locking script. This is mostly used in coinbase transactions which was generated by mining software which is not up to date.
\par

The following in an example of a P2PK locking script:

\begin{lstlisting}[basicstyle=\ttfamily]
<Public Key A> OP_CHECKSIG
\end{lstlisting}

To unlock this, the following script must be submitted:
\begin{lstlisting}[basicstyle=\ttfamily]
<Signature from Private Key A>
\end{lstlisting}

Together, they form the script which is validated transaction validation software. The script becomes:

\begin{lstlisting}[basicstyle=\ttfamily]
<Signature from Private Key A> <Public Key A> OP_CHECKSIG
\end{lstlisting}

This script uses the CHECKSIG operater. This operator returns TRUE on the stack if the signature belongs to the correct key. \cite{book:1317587}
% subsection overview (end)

\subsection{Problems} % (fold)
\label{sub:problems}
The P2PK is longer than P2PKH discussed in section \ref{sec:pay_to_public_key_hash_p2pkh}. This is because for people to make transactions, the would have to pass the whole publik key instead of the address, which is much longer. \cite{walker}
\par
Apart from that, P2PK is also more unsafe than P2PKH. Bitcoin uses the Elliptic Curve Digital Signature Algorithm (ECDA) signature scheme. This means the the bitcoins that use P2PK are only safe until the ECDA is cracked. An algorithm already exists that cracks the ECDA and is called Shor's Algorithm. However, to work effectively, this algorithm needs a quantum computer. 

% subsection problems (end)

% section pay_to_pubkey_p2pk (end)
\section{Pay-to-Public-Key-Hash (P2PKH)} % (fold)
\label{sec:pay_to_public_key_hash_p2pkh}

This script is used for most of the transactions processed on the bitcoin network. Within this script, the locking script encrypts the output with a public key hash. This is known as a bitcoin address. When an output is spent, the lock on the output by a P2PKH is removed. This unlocking is done by handing over a public key and a digital signature that were created by the corresponding private key.
\par
An example of a locking script is: 
\begin{lstlisting}[basicstyle=\ttfamily]
OP_DUP OP_HASH160 <Cafe Public Key Hash> OP_EQUAL OP_CHECKSIG
\end{lstlisting}

The corresponding unlocking script looks like:
\begin{lstlisting}[basicstyle=\ttfamily]
<Cafe Signature> <Cafe Public Key>
\end{lstlisting}

The validation scripts of the previous two scripts looks like:
\begin{lstlisting}[basicstyle=\ttfamily]
<Cafe Signature> <Cafe Public Key> OP_DUP OP_HASH160 <Cafe Public Key Hash> OP_EQUAL OP_CHECKSIG
\end{lstlisting}


If the unlocking script has a valid signature that corresponds to the publid key hash set as an encumbrance, the script returns TRUE on the stack. \cite{book:1317587}

% section pay_to_public_key_hash_p2pkh (end)


\section{Bitcoin Script Example 1}
\paragraph{Question}
If you were given, 767695935687, which is a binary encoding for a Bitcoin script ScriptPubKey, then what ScriptSig would you need to combine the ScriptPubSig with to execute or unlock the ScriptPubKey?
Hint: 76 → OP\_DUP
\paragraph{Solution}
Changing the assembly \texttt{767695935687} to bitcoin script we get \texttt{OP\_DUP OP\_DUP OP\_MUL OP\_ADD OP\_6 OP\_EQUAL}.
\par
When we translate this to normal algebra we get $(x*x)+x=6$. Making $x$ subject of the formula:
$$x^2 + x - 6 = 0$$
$$(x + 3)(x - 2) = 0$$
$$ x = -3, x = 2$$

Therefore there are two signatures that will release cash and these are: \texttt{OP\_2} and \texttt{OP\_3 OP\_NEGATE}

\section{Bitcoin Script Example 2}
\paragraph{Question}
Figure out what this script is doing: \texttt{6e879169a77ca787}

\paragraph{Solution}
Translating the assembly \texttt{} to bitcoin script, we get: \texttt{OP\_2DUP OP\_EQUAL OP\_NOT OP\_VERIFY OP\_SHA1 OP\_SWAP OP\_SHA1 OP\_EQUAL}. 

\begin{center}
	\resizebox{\textwidth}{!}{\begin{tabular}{ |c| c| }
		\hline
		\textbf{Word} & \textbf{Description} \\ 
		\hline
		OP\_2DUP & Duplicates the top two stack items.  \\  
		\hline
		OP\_EQUAL & Returns 1 if the inputs are exactly equal, 0 otherwise. \\
		\hline
		OP\_NOT & If the input is 0 or 1, it is flipped. Otherwise the output will be 0. \\
		\hline
		OP\_VERIFY & Marks transaction as invalid if top stack value is not true. The top stack value is removed.  \\
		\hline
		OP\_SHA1 & The input is hashed using SHA-1. \\
		\hline
		OP\_SWAP & The top two items on the stack are swapped. \\
		\hline
		OP\_SHA1 &  The input is hashed using SHA-1. \\
		\hline
		OP\_EQUAL & Returns 1 if the inputs are exactly equal, 0 otherwise. \\
		\hline
	\end{tabular}}
\end{center}

This script returns TRUE if a hash collision is found and FALSE otherwise.
\cite{bitcoinscript}


\bibliographystyle{abbrv}
	
 \bibliography{references}

\end{document}
